\documentclass[11pt,svgnames,smaller,aspectratio=169,italian]{beamer}

\usepackage[utf8]{inputenc}
\usepackage{amsmath}
\usepackage[italian]{babel}


\usetheme{UniPi}

% ------------------ CHANGE FROM HERE ------------------
\title{Crittografia a chiave pubblica: uno sguardo alle vulnerabilità \\ di RSA e Diffie-Hellman}
\author{Leonardo Alfreducci}
\institute[Università di Pisa]{Università di Pisa\\Dipartimento di Informatica}
\date{Pisa, 7 ottobre 2022}
% extra can be used for thesis supervisors and similar
\extra{Relatori\\Dott. Gaspare Ferraro\\Prof.ssa Anna Bernasconi}
% ------------------ END CHANGE ------------------

\usepackage{transparent}
\mode<presentation>

\begin{document}

\begin{frame} 
	\titlepage
\end{frame}
\logo{\transparent{0.2}\includegraphics[height=2cm]{images/cherubino}}



% ------------------ CHANGE FROM HERE ------------------
\begin{frame}{Indice}
	\tableofcontents
\end{frame}

%%%%%%%%%%%%%%%%%%%%%%%%%%%%%%%%%%%%%%%%%%%%%%%%%%%%%%%%
\section{Introduzione}
\begin{frame}
	\sectionpage
	\centering
\end{frame}

\begin{frame}{Introduzione}
	\begin{itemize}
		\item Lo scambio di informazioni di ogni tipo avviene attraverso la rete: è dunque di fondamentale importanza proteggere le informazioni che vengono scambiate.
		\item Si passeranno in rassegna i due protocolli più usati per lo scambio di chiave: RSA e Diffie-Hellman, quest'ultimo analizzato su campo primo e su curve ellittiche.
		\item Lo scopo della tesi è quello di andare al di là di una trattazione teorica di questi due protocolli, concentrandosi piuttosto sull'aspetto pratico.
	\end{itemize}
\end{frame}

%%%%%%%%%%%%%%%%%%%%%%%%%%%%%%%%%%%%%%%%%%%%%%%%%%%%%%%%
\section{RSA} %da rinominare
\begin{frame}
	\sectionpage
	\centering
\end{frame}

\begin{frame}{La teoria di RSA} %da rinominare
	\begin{itemize}
		\item È un cifrario asimmetrico. Sono dunque presenti due coppie di chiavi:
		\begin{itemize}
			\item $(e, n)$ utilizzata per cifrare (\emph{chiave pubblica});
			\item $(d, n)$ utilizzata per decifrare (\emph{chiave privata}).
		\end{itemize}
		\item Si scelgono due numeri primi $p$ e $q$.
		\item Si calcola $n = p \cdot q$ e $\phi(n) = (p - 1) \cdot (q - 1)$.
		\item Si sceglie $e < \phi(n)$ tale che $gcd(e, n) = 1$.
		\item Si calcola $d = e^{-1} \mod \phi(n)$.
		\item Tutti i passi descritti possono essere svolti in tempo polinomiale.
	\end{itemize}
\end{frame}

\begin{frame}{RSA: cifratura e decifrazione}
	\begin{itemize}
		\item Per cifrare un messaggio $m$ è sufficiente calcolare il crittogramma $c$ come:
		\begin{equation*}
			c = m^{e} \mod n.
		\end{equation*}
		\item Per ottenere il messaggio $m$ dato $c$ è sufficiente calcolarlo come:
		\begin{equation*}
			m = c^{d} \mod n.
		\end{equation*}
		
	\end{itemize}
	
\end{frame}

\begin{frame}{La sicurezza di RSA 1} %da riniominare?
	\begin{itemize}
		\item La sicurezza di RSA è garantita grazie al problema della fattorizzazione di un numero $n$ come prodotto di due fattori $p \cdot q$.
		\item Per questo è importante scegliere due fattori primi molto grandi, tale che il modulo sia almeno $2048$ bit, meglio ancora se $3072$ bit.
		\item Nel 1999 è stato fattorizzato RSA-512 in circa $7$ mesi utilizzando centinaia di calcolatori e impiegando l'equivalente di 8400 anni di CPU.
			\begin{itemize}
				\item Nel 2009 lo stesso attacco poteva essere effettuato in 83 giorni da un solo calcolatore.
			\end{itemize}
		\item Nel 2020 il numero più grande fattorizzato ha 829 bit, impiegando l'equivalente di 2700 anni di CPU.
	\end{itemize}
\end{frame}

\begin{frame}{La sicurezza di RSA 2} %da riniominare?
	\begin{itemize}
		\item Ho implementato tre algoritmi per la fattorizzazione:
			\begin{itemize}
				\item \emph{Wheel factorization}: fondamentalmente un brute force sul numero, cercando i divisori;
				\item \emph{Pollard's rho factorization}: di natura probabilistica, è quello più efficiente;
				\item \emph{Fermat factorization}: è più efficiente se i due numeri primi sono vicini tra loro.
			\end{itemize}
		\item Ho fattorizzato 120 bit utilizzando l'algoritmo \emph{Pollard's rho} in poco meno di un'ora sul mio calcolatore.
	\end{itemize}
\end{frame}

\begin{frame}{}

\end{frame}






















\end{document}
