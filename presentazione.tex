\documentclass[11pt,svgnames,smaller,aspectratio=169,italian]{beamer}

\usepackage[utf8]{inputenc}
\usepackage{amsmath}
\usepackage[italian]{babel}
\usepackage{dsfont}
\usepackage{multirow}
%\usepackage{tikz-cd,amsmath}


\newcommand{\extrabegin}{
   \newcounter{finalframe}
   \setcounter{finalframe}{\value{framenumber}}
}
\newcommand{\extraend}{
   \setcounter{framenumber}{\value{finalframe}}
}

\usetheme{UniPi}

%-------------------TABLE------------------------------
\usepackage{siunitx} %tabelle valore S
\newcolumntype{P}[1]{>{\centering\arraybackslash}p{#1}}
\newcolumntype{M}[1]{>{\centering\arraybackslash}m{#1}}

% ------------------ CHANGE FROM HERE ------------------
\title{Crittografia a chiave pubblica: uno sguardo alle vulnerabilità \\ di RSA e Diffie-Hellman}
\author{Leonardo Alfreducci}
\institute[Università di Pisa]{Università di Pisa\\Dipartimento di Informatica}
\date{Pisa, 7 ottobre 2022}
% extra can be used for thesis supervisors and similar
\extra{Relatori\\Dott. Gaspare Ferraro\\Prof.ssa Anna Bernasconi}
% ------------------ END CHANGE ------------------

\usepackage{transparent}
\mode<presentation>

\begin{document}

\begin{frame} 
	\titlepage
\end{frame}
\logo{\transparent{0.2}\includegraphics[height=2cm]{images/cherubino}}



% ------------------ CHANGE FROM HERE ------------------
\begin{frame}{Indice}
	\tableofcontents
\end{frame}

%%%%%%%%%%%%%%%%%%%%%%%%%%%%%%%%%%%%%%%%%%%%%%%%%%%%%%%%
\section{Introduzione}
\begin{frame}
	\sectionpage
	\centering
\end{frame}

\begin{frame}{Introduzione}
	\begin{itemize}
		\item Una grandissima quantità di informazioni viaggia attraverso la rete: è dunque di fondamentale importanza proteggere i dati che vengono scambiati.
		\item Si passeranno in rassegna i due protocolli più usati per lo scambio di chiave: RSA e Diffie-Hellman, quest'ultimo analizzato su campo primo e su curve ellittiche.
		\item Lo scopo della tesi è quello di andare al di là di una trattazione teorica di questi due protocolli, concentrandosi piuttosto sull'aspetto pratico.
	\end{itemize}
\end{frame}

%%%%%%%%%%%%%%%%%%%%%%%%%%%%%%%%%%%%%%%%%%%%%%%%%%%%%%%%
\section{RSA}
\begin{frame}
	\sectionpage
	\centering
\end{frame}

\begin{frame}{RSA: la teoria dietro al protocollo}
	\begin{itemize}
		\item È un cifrario asimmetrico. È dunque presente una coppia di chiavi:
		\begin{itemize}
			\item $(e, n)$ utilizzata per cifrare (\emph{chiave pubblica});
			\item $(d, n)$ utilizzata per decifrare (\emph{chiave privata}).
		\end{itemize}
		\item Si scelgono due numeri primi $p$ e $q$.
		\item Si calcola $n = p \cdot q$ e $\phi(n) = (p - 1) \cdot (q - 1)$.
		\item Si sceglie $e < \phi(n)$ tale che $gcd(e, n) = 1$.
		\item Si calcola $d = e^{-1} \mod \phi(n)$.
		\item Tutte le operazioni descritte possono essere svolte in tempo polinomiale.
	\end{itemize}
\end{frame}

\begin{frame}{RSA: cifratura e decifratura}
	\begin{itemize}
		\item Per cifrare un messaggio $m$ è sufficiente calcolare il crittogramma $c$ come:
		\begin{equation*}
			c = m^{e} \mod n.
		\end{equation*}
		\item Per ottenere il messaggio $m$ dato $c$ è sufficiente calcolarlo come:
		\begin{equation*}
			m = c^{d} \mod n.
		\end{equation*}
		
	\end{itemize}
	
\end{frame}

\begin{frame}{RSA: uno sguardo alla sicurezza}
	\begin{itemize}
		\item La sicurezza di RSA è garantita grazie al problema della fattorizzazione di un numero $n$ come prodotto di due fattori $p \cdot q$.
		\item Per questo è importante scegliere due fattori primi molto grandi, tale che il modulo sia almeno $2048$ bit, meglio ancora se $3072$ bit.
		\item Nel 1999 è stato fattorizzato RSA-512 in circa $7$ mesi utilizzando centinaia di calcolatori e impiegando l'equivalente di 8400 anni di CPU.
			\begin{itemize}
				\item Nel 2009 lo stesso attacco poteva essere effettuato in 83 giorni da un solo calcolatore.
			\end{itemize}
		\item Nel 2020 il numero più grande fattorizzato ha 829 bit, impiegando l'equivalente di 2700 anni di CPU.
	\end{itemize}
\end{frame}

\begin{frame}{RSA: l'esponente pubblico $e$}
	\begin{itemize}
		\item L'esponente pubblico, dato che non contiene alcuna informazione, viene generalmente riutilizzato per molteplici operazioni.
	\end{itemize}
	\begin{table}[]
		\centering
		\begin{tabular}{  M{1cm} S[table-format=2.2, table-space-text-post = \si{\meter}] | M{1.3cm}  S[table-format=2.2, table-space-text-post = \si{\meter}]  | M{1cm}  S[table-format=2.2, table-space-text-post = \si{\meter}] } 
			\multicolumn{2}{c|}{\emph{X.509}} & \multicolumn{2}{c|}{\emph{PGP}} & \multicolumn{2}{c}{\emph{Combinati}} \\ \hline
			\multicolumn{1}{c}{$e$}       	&\multicolumn{1}{c |}{$\%$}           & \multicolumn{1}{c}{$e$}             & \multicolumn{1}{c |}{$\%$}      & \multicolumn{1}{c}{$e$}         & \multicolumn{1}{c}{$\%$}             \\ \hline
			65537       & 98.4921     & 65537          	& 48.8501 	& 65537  		& 95.4933        \\
			17          	& 0.7633       & 17             		& 39.5027 	& 17 			& 3.1035         \\
			3           	& 0.3772       & 41             		& 7.5727  		& 41           	& 0.4574         \\
			35          	& 0.1410       & 19             		& 2.4774  		& 3            	& 0.3578         \\
			altri       	& 0.2264       & altri          		& 1.6271  		& altri        	& 0.588        
		\end{tabular}
		%prima colonna dopo il 35: 100-(98,4921+0,7633+0,3772+0,1410) = 0,2264
		%seconda colonna dopo il 19: 100-(48,8501+39,5027+7,5727+2,4474) = 1,6271
		%terza colonna dopo il 3: 100-(95,4933+3,1035+0,4574+0,3578) = 0,588
	\end{table}
	\begin{itemize}
		\item Un bug di un'implementazione di Python SaltStack imponeva $e = 1$.
		\item Si deve prestare attenzione che intervenga la riduzione in modulo.
	\end{itemize}
\end{frame}

\begin{frame}{RSA: malleabilità}
	\begin{itemize}
		\item RSA è \emph{malleabile}.
			\begin{itemize}
				\item Ad esempio, se un attaccante conosce $c = m^{e} \mod n$, può sostituire $c' = c \cdot 2^{e} \mod n$.
				\item Quando $c'$ verrà decifrato, si otterrà $2m$ invece che l'originario $m$.
			\end{itemize}
		\item Con il padding questa modifica molto semplice non è più possibile.
	\end{itemize}
\end{frame}

\begin{frame}{RSA: gli schemi di padding}
	\begin{itemize}
		\item Prima di essere cifrato mediante RSA, ogni messaggio viene modificato con gli schemi di padding.
		\item Gli schemi di padding sono importanti in crittografia:
			\begin{itemize}
				\item aggiungono una componente di casualità;
				\item non rendono possibile un recupero anche parziale del messaggio, fissandone univocamente la lunghezza.
			\end{itemize}
	\end{itemize}
\end{frame}

\begin{frame}{RSA: generazione errata della chiave}
	\begin{itemize}
		\item L'esponente $e$ deve essere scelto coprimo con $\phi(n)$.
		\item In una pre-release di Windows 10, nel 2019, non veniva effettuato il controllo che $gcd(e,\phi(n)) = 1$ nel momento in cui veniva scelto l'esponente pubblico.
		\item Il corretto funzionamento di RSA è compromesso e la decifratura non è più possibile.
	\end{itemize}
\end{frame}

\begin{frame}{RSA: la probabilità di scegliere l'esponente pubblico errato}
	\begin{itemize}
		\item Ma quanto spesso questo problema si verifica nella pratica?
		\item Per $e = 65537$, la probabilità $P$ che $e$ venga scelto in modo errato è data da
			\begin{equation*}
				P < \frac{1}{32000}
			\end{equation*}
		\item Windows 10 è utilizzato da oltre un miliardo di dispositivi.
			\begin{itemize}
				\item Più di 30000 utenti coinvolti.
			\end{itemize}
	\end{itemize}
\end{frame}

\begin{frame}{RSA: il recupero dei messaggi erroneamente cifrati}
	\begin{itemize}
		\item Se la chiave viene generata in modo errato ad ogni crittogramma potrebbero corrispondere $e$ messaggi che lo generino.
		\item Se i messaggi perduti sono importanti?
		\item Il recupero dei messaggi è esponenziale nella dimensione di $e$.
			\begin{itemize}
				\item Per fortuna, $e$ viene generalmente scelto basso.
				\item Per $e = 65537$ il recupero dei messaggi con un moderno calcolatore avviene in circa 30 secondi.
			\end{itemize}
		\item Per scartare i messaggi errati, si possono utilizzare gli schemi di padding.
	\end{itemize}
\end{frame}

\begin{frame}{RSA: moduli ripetuti}
	\begin{itemize}
		\item È comune che uno stesso modulo $n$ sia condiviso tra più host.
			\begin{itemize}
				\item Il 4\% dei moduli usati in HTTPS risulta condiviso tra più host.
				\item Il 60\% delle chiavi SSH e il 65\% di quelle usate per IPv4 risultano condivise.
				\item Non è una vulnerabilità se gli host non sono correlati.
			\end{itemize}
		\item Nel 2013, molti router e dispositivi della stessa linea di un produttore condividevano lo stesso modulo: si potevano decifrare i testi a vicenda.
	\end{itemize}
\end{frame}

%%%%%%%%%%%%%%%%%%%%%%%%%%%%%%%%%%%%%%%%%%%%%%%%%%%%%%%%
\section{Diffie-Hellman su campo primo}
\begin{frame}
	\sectionpage
	\centering
\end{frame}

\begin{frame}{DH su campo primo: la teoria dietro al protocollo}
	\begin{itemize}
		\item È un protocollo per lo scambio di chiave.
		\item $A$ e $B$, che vogliono comunicare, si devono accordare su due parametri:
			\begin{itemize}
				\item un numero primo $p$;
				\item un generatore $g$ di $\mathds{Z}_{p}^{*}$.
			\end{itemize}
		\item Per lo scambio di chiave:
			\begin{itemize}
				\item $A$ e $B$ scelgono due interi casuali $a, b$ tali che $1 < a, b < p - 1$;
				\item $A$ calcola $n_{A} = g^{a} \mod p$, $B$ calcola $n_{B} = g^{b} \mod p$;
				\item $A$ e $B$ si scambiano rispettivamente $n_{A}$ ed $n_{B}$;
				\item $A$ calcola $k = n_{B}^{a} \mod p$, $B$ calcola $k = n_{A}^{b} \mod p$;
				\item $k$ è condivisa poiché $k = g^{a \cdot b} \mod p$.
			\end{itemize}
	\end{itemize}
\end{frame}

\begin{frame}{DH su campo primo: uno sguardo alla sicurezza}
	\begin{itemize}
		\item Un attaccante per decifrare la comunicazione dovrebbe calcolare il logaritmo discreto:
			\begin{equation*}
				a = \log_{g} n_{A}	\quad\text{\emph{oppure}}	\quad b = \log_{g} n_{B}.
			\end{equation*}
			\begin{itemize}
				\item Ad oggi non sono noti algoritmi eseguire questa operazione in tempo polinomiale.
				\item L'attacco migliore è l'\emph{Index calculus}, che ha complessità $O(2^{\sqrt{b \cdot \ln b}})$ per chiavi di $b$ bit.
			\end{itemize}
		\item $p$ dal 2013 dovrebbe essere almeno di 2048 bit.
			\begin{itemize}
				\item Il calcolo del logaritmo discreto con $p$ a 530 bit è stato effettuato nel 2007.
			\end{itemize}
	\end{itemize}
\end{frame}

\begin{frame}{DH su campo primo: moduli non primi}
	\begin{itemize}
		\item In alcune implementazioni $p$ non è un primo.
		\item È una vulnerabilità solo se $p$ ha fattori primi molto piccoli.
			\begin{itemize}
				\item Si potrebbe usare il \emph{Teorema cinese del resto}.
				\item Il calcolo del logaritmo discreto si presume sia difficile tanto quanto il problema della fattorizzazione.
			\end{itemize}
	\end{itemize}
\end{frame}

\begin{frame}{DH su campo primo: valori da evitare}
	\begin{itemize}
		\item Sottogruppi di ordine $1$ e $2$ sono da evitare come scambi di chiave.
		\item Se $g$ genera l'intero gruppo, allora sono presenti i sottogruppi di ordine $1$ e $2$.
			\begin{itemize}
				\item Se viene scambiato il valore $1$, il segreto condiviso non può che essere $1$.
				\item Più grave il caso in cui venga scambiato il valore $-1$.
				%\item Se i primi sono generati come spiegato precedentemente per essere \emph{sicuri} ($p = 2q + 1$) allora non contengono questi valori.
			\end{itemize}
	\end{itemize}
\end{frame}

\begin{frame}{DH su campo primo: un bit insicuro}
	\begin{itemize}
		\item Con il sottogruppo di ordine $2$ la situazione è più grave.
			\begin{itemize}
				\item $B$ è un host che ricicla la stessa chiave Diffie-Hellman per molteplici connessioni.
				\item $A$ (l'attaccante) manda a $B$ il valore di scambio $n_{A} = -1$.
				\item La chiave condivisa non può che essere $1$ o $-1$.
				\item Avendo modo di verificare quale delle due sia la chiave risultante, $A$ può capire se il segreto di $B$ è un numero pari o dispari, facendo perdere così un bit di sicurezza a $B$.
			\end{itemize}
		\item Nel 2016 il 3\% dei server HTTPS e il 34\% dei server SSH accettava $-1$ come valore di scambio.
		%\item Se i primi i parametri una determinata struttura, ovvero $p = 2q + 1$, e $g$ genera un sottogruppo di ordine $q$, questo non accade.
		\item I parametri di gruppo ritenuti \emph{sicuri}, ovvero con primi del tipo $p = 2q + 1$ e con $g$ che genera sottogruppi di ordine $q$, comprendono solo valori di ordine elevato.
	\end{itemize}
\end{frame}

%%%%%%%%%%%%%%%%%%%%%%%%%%%%%%%%%%%%%%%%%%%%%%%%%%%%%%%%
\section{Diffie-Hellman su curve ellittiche}
\begin{frame}
	\sectionpage
	\centering
\end{frame}

\begin{frame}{Crittografia su curve ellittiche: una panoramica}
	\begin{itemize}
		\item La crittografia su curve ellittiche, a parità di sicurezza, richiede chiavi di molti meno bit.
			\begin{itemize}
				\item Di conseguenza, le operazioni sono più veloci.
			\end{itemize}
		\item Si comparano nella seguente tabella i bit necessari a parità di sicurezza nei tre protocolli descritti.
	\end{itemize}
	\begin{table}
		\centering
		\begin{tabular}{c|c}
			\begin{tabular}[c]{@{}c@{}}RSA e DH\\ (bit del modulo)\end{tabular} & \begin{tabular}[c]{@{}c@{}}ECC\\ (bit dell'ordine)\end{tabular}  \\ 
			\hline
			1024                                                                & 160                                                              \\
			2048                                                                & 224                                                              \\
			3072                                                                & 256                                                              \\
			7680                                                                & 384                                                              \\
			15360                                                              & 512                                                             
		\end{tabular}
	\end{table}
\end{frame}

\begin{frame}{Crittografia su curve ellittiche: il protocollo Diffie-Hellman}
	\begin{itemize}
		\item Diffie-Hellman si presta molto bene all'applicazione su curve ellittiche.
		\item È stato adottato su larga scala negli ultimi 10 anni.
		\item Ad oggi è utilizzato da più del 65\% degli scambi di chiave.
	\end{itemize}
\end{frame}

\begin{frame}{DH su curve ellittiche: la teoria dietro al protocollo}
	\begin{itemize}
		\item $A$ e $B$, che vogliono comunicare, si devono accordare su due parametri:
			\begin{itemize}
				\item una curva ellittica prima $E_{p}(a, b) = \{(x, y) \in \mathds{Z}_{p}^{2} | y^{2} \mod p = (x^{3} + a \cdot x + b) \mod p \} \cup \{0\}$;
				\item un generatore $G$ primo facente parte di un sottogruppo della curva di ordine $n$.
			\end{itemize}
		\item Per lo scambio di chiave:
			\begin{itemize}
				\item $A$ e $B$ scelgono due interi casuali $a, b$ tali che $a, b < n$;
				\item $A$ calcola $P_{A} = a \cdot G$, $B$ calcola $P_{B} = b \cdot G$;
				\item $A$ e $B$ si scambiano rispettivamente $P_{A}$ ed $P_{B}$;
				\item $A$ calcola $k = a \cdot P_{B}$, $B$ calcola $k = b \cdot P_{A}$;
				\item $k$ è condivisa poiché $k = n_{A} \cdot n_{B} \cdot G$.
			\end{itemize}
	\end{itemize}
\end{frame}

\begin{frame}{DH su curve ellittiche: uno sguardo alla sicurezza}
	\begin{itemize}
		\item La sicurezza è basata sulla difficoltà di calcolare, dati due punti $P$ e $Q$, il più piccolo intero $k$ tale che $P = k \cdot Q$.
			\begin{itemize}
				\item Chiamato \emph{problema della risoluzione del logaritmo discreto su curve ellittiche}.
				\item Il migliore attacco ad oggi è \emph{Pollard rho} che ha complessità $O(2^{b/2})$.
			\end{itemize}
		\item Si fa affidamento a curve sicure, scelte da una lista del NIST (National Institute of Standards and Technology).
	\end{itemize}
\end{frame}

\begin{frame}{DH su curve ellittiche: i parametri della curva \emph{P-384}}
	\begin{itemize}
		\item È una delle curve ellittiche più utilizzate e raccomandate.
		\item $p = 2^{284} - 2^{128} - 2^{96} + 2^{32} - 1$;
		\item $a = -3$;
		\item $b =  27580193559959705877849011840389048093056905856361568521$\textbackslash \\
		$428707301988689241309860865136260764883745107765439761230575$, avente la proprietà che $2^{383} < b < 2^{384}$;
		 \item $N \simeq 2^{384} - 2^{190}$, che rappresenta l'\emph{ordine della curva}, ovvero il numero di punti che ammette al suo interno.
	\end{itemize}
\end{frame}

\begin{frame}{DH su curve ellittiche: attacco con curva invalida}
	\begin{itemize}
		\item Attacco simile a quello descritto per DH su campo primo.
		\item Se il valore inviato non giace sulla curva ed ha un ordine basso $q_{i}$, un attaccante può calcolare il segreto della vittima modulo $q_{i}$.
		\item Se la vittima ricicla il suo segreto per molteplici connessioni, effettuando questo attacco più volte si può recuperare l'intero segreto.
		\item Si dovrebbe controllare che il punto inviato giaccia sulla curva corretta.
			\begin{itemize}
				\item Nel 2015 tre di otto popolari librerie TLS non effettuavano questo controllo.
				\item Si è stimato che nel 2015 lo 0,8\% dei siti HTTPS non effettuavano questo controllo.
			\end{itemize}
	\end{itemize}
\end{frame}

%%%%%%%%%%%%%%%%%%%%%%%%%%%%%%%%%%%%%%%%%%%%%%%%%%%%%%%%
\section{Conclusioni}
\begin{frame}
	\sectionpage
	\centering
\end{frame}

\begin{frame}{Conclusioni}
	\begin{itemize}
		\item RSA è molto più soggetto ad errori di implementazione e per questo si cerca di non utilizzarlo.
		\item La strada è sempre più verso le curve ellittiche.
		\item I protocolli presentati non sono sicuri per utilizzi post-quantistici.
			\begin{itemize}
				\item L'\emph{Algoritmo di Shor} sui computer quantistici permette di calcolare la fattorizzazione e il logaritmo discreto per le curve ellittiche in tempo polinomiale probabilistico.
			\end{itemize}
		\item La crittografia è in continuo studio ed evoluzione: ciò che oggi è considerato sicuro domani potrebbe non esserlo.
			\begin{itemize}
				\item La fattorizzazione e il calcolo del logaritmo discreto sono problemi che ad oggi non conoscono algoritmi polinomiali per il loro calcolo, ma potrebbero esistere.
				\item È importante e fondamentale restare aggiornati con gli studi.
			\end{itemize}
		\item A luglio 2022 il NIST ha pubblicato quattro algoritmi resistenti ai computer quantistici.
			\begin{itemize}
				\item Un mese dopo uno di questi algoritmi è stato violato utilizzando un classico calcolatore.
			\end{itemize}
	\end{itemize}
\end{frame}

%%%%%%%%%%%%%%%%%%%%%%%%%%%%%%%%%%%%%%%%%%%%%%%%%%%%%%%%
\begin{frame}
		\frametitle{Fine}
		\centering
		\Large
		Grazie per l'attenzione!
\end{frame}

%%%%%%%%%%%%%%%%%%%%%%%%%%%%%%%%%%%%%%%%%%%%%%%%%%%%%%%%
\extrabegin

\section*{Extra}
\begin{frame}
	\sectionpage
	\centering
\end{frame}

\begin{frame}{Extra: attacco con curva invalida}
	\begin{itemize}
		\item Si supponga che $A$ sia l'attaccante e $B$ la vittima.
		\item Si supponga che $A$ mandi il punto $P_{A}$ come valore di scambio, avente un basso ordine 5.
			\begin{itemize}
				\item Questo significa che $5 \cdot P_{A} = O$.
			\end{itemize}
		\item La chiave $k$ non potrà che assumere uno tra i seguenti valori:
			\begin{equation*}
				1 \cdot P_{A} \quad\text{\emph{}}\quad 2 \cdot P_{A} \quad\text{\emph{}}\quad 3 \cdot P_{A} \quad\text{\emph{}}\quad 4 \cdot P_{A} \quad\text{\emph{}}\quad 5 \cdot P_{A} = O.
			\end{equation*}
		\item Si supponga che $k = 3 \cdot P_{A}$.
		\item Allora $n_{B}$ avrà necessariamente la forma $n_{B} = 3 + 5 \cdot l$, per qualche $l$.
			\begin{itemize}
				\item La chiave $k$ può essere scritta come
					\begin{equation*}
						k = (3 + 5 \cdot l) \cdot P_{A}  = 3 \cdot P_{A} + 5 \cdot P_{A} \cdot l = 3 \cdot P_{A}.
					\end{equation*}
			\end{itemize}
		\item $A$ è riuscito a calcolare $n_{B} \mod 5 = 3$.
	\end{itemize}
\end{frame}

\begin{frame}{}
	\begin{itemize}
		\item 
	\end{itemize}
\end{frame}

\extraend







% INTRODUZIONE:
% - una grande quantità di dati passa attraverso la rete
% crittografia simmetrica
% - è importante proteggerli in modo efficiente
% - si passano in rassegna due protocolli
% - scopo della tesi: aspetto implementativo


% togliere scartare messaggi durante il recupero e mettere solo un item



%DIFFIE-HELLMA SU CAMPO PRIMO
% primi sicuri perché generano sottogruppi più grandi dove è possibile scegliere il segreto in un range più ampio, brute force più difficile




% DIFFIE-HELLMAN SU CURVE ELLITTICHE
% ricordarsi il discorso della difficoltà di calcolare l'ordine di un punto per l'invalid curve attack


\end{document}
