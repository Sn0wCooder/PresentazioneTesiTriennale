\documentclass[11pt,svgnames,smaller,aspectratio=169,italian]{beamer}

\usepackage[utf8]{inputenc}
\usepackage{amsmath}
\usepackage[italian]{babel}


\usetheme{UniPi}

%-------------------TABLE------------------------------
\usepackage{siunitx} %tabelle valore S
\newcolumntype{P}[1]{>{\centering\arraybackslash}p{#1}}
\newcolumntype{M}[1]{>{\centering\arraybackslash}m{#1}}

% ------------------ CHANGE FROM HERE ------------------
\title{Crittografia a chiave pubblica: uno sguardo alle vulnerabilità \\ di RSA e Diffie-Hellman}
\author{Leonardo Alfreducci}
\institute[Università di Pisa]{Università di Pisa\\Dipartimento di Informatica}
\date{Pisa, 7 ottobre 2022}
% extra can be used for thesis supervisors and similar
\extra{Relatori\\Dott. Gaspare Ferraro\\Prof.ssa Anna Bernasconi}
% ------------------ END CHANGE ------------------

\usepackage{transparent}
\mode<presentation>

\begin{document}

\begin{frame} 
	\titlepage
\end{frame}
\logo{\transparent{0.2}\includegraphics[height=2cm]{images/cherubino}}



% ------------------ CHANGE FROM HERE ------------------
\begin{frame}{Indice}
	\tableofcontents
\end{frame}

%%%%%%%%%%%%%%%%%%%%%%%%%%%%%%%%%%%%%%%%%%%%%%%%%%%%%%%%
\section{Introduzione}
\begin{frame}
	\sectionpage
	\centering
\end{frame}

\begin{frame}{Introduzione}
	\begin{itemize}
		\item Una grandissima quantità di informazioni viaggia attraverso la rete: è dunque di fondamentale importanza proteggere i dati che vengono scambiati.
		\item Si passeranno in rassegna i due protocolli più usati per lo scambio di chiave: RSA e Diffie-Hellman, quest'ultimo analizzato su campo primo e su curve ellittiche.
		\item Lo scopo della tesi è quello di andare al di là di una trattazione teorica di questi due protocolli, concentrandosi piuttosto sull'aspetto pratico.
	\end{itemize}
\end{frame}

%%%%%%%%%%%%%%%%%%%%%%%%%%%%%%%%%%%%%%%%%%%%%%%%%%%%%%%%
\section{RSA} %da rinominare
\begin{frame}
	\sectionpage
	\centering
\end{frame}

\begin{frame}{La teoria di RSA} %da rinominare
	\begin{itemize}
		\item È un cifrario asimmetrico. Sono dunque presenti due coppie di chiavi:
		\begin{itemize}
			\item $(e, n)$ utilizzata per cifrare (\emph{chiave pubblica});
			\item $(d, n)$ utilizzata per decifrare (\emph{chiave privata}).
		\end{itemize}
		\item Si scelgono due numeri primi $p$ e $q$.
		\item Si calcola $n = p \cdot q$ e $\phi(n) = (p - 1) \cdot (q - 1)$.
		\item Si sceglie $e < \phi(n)$ tale che $gcd(e, n) = 1$.
		\item Si calcola $d = e^{-1} \mod \phi(n)$.
		\item Tutti i passi descritti possono essere svolti in tempo polinomiale.
	\end{itemize}
\end{frame}

\begin{frame}{RSA: cifratura e decifrazione}
	\begin{itemize}
		\item Per cifrare un messaggio $m$ è sufficiente calcolare il crittogramma $c$ come:
		\begin{equation*}
			c = m^{e} \mod n.
		\end{equation*}
		\item Per ottenere il messaggio $m$ dato $c$ è sufficiente calcolarlo come:
		\begin{equation*}
			m = c^{d} \mod n.
		\end{equation*}
		
	\end{itemize}
	
\end{frame}

\begin{frame}{La sicurezza di RSA 1} %da riniominare?
	\begin{itemize}
		\item La sicurezza di RSA è garantita grazie al problema della fattorizzazione di un numero $n$ come prodotto di due fattori $p \cdot q$.
		\item Per questo è importante scegliere due fattori primi molto grandi, tale che il modulo sia almeno $2048$ bit, meglio ancora se $3072$ bit.
		\item Nel 1999 è stato fattorizzato RSA-512 in circa $7$ mesi utilizzando centinaia di calcolatori e impiegando l'equivalente di 8400 anni di CPU.
			\begin{itemize}
				\item Nel 2009 lo stesso attacco poteva essere effettuato in 83 giorni da un solo calcolatore.
			\end{itemize}
		\item Nel 2020 il numero più grande fattorizzato ha 829 bit, impiegando l'equivalente di 2700 anni di CPU.
	\end{itemize}
\end{frame}

\begin{frame}{La sicurezza di RSA 2} %da riniominare?
	\begin{itemize}
		\item Sono stati implementati tre algoritmi per la fattorizzazione:
			\begin{itemize}
				\item \emph{Wheel factorization}: fondamentalmente un brute force sul numero, cercando i divisori;
				\item \emph{Pollard's rho factorization}: di natura probabilistica, è quello più efficiente;
				\item \emph{Fermat factorization}: è più veloce se i due numeri primi sono vicini tra loro.
			\end{itemize}
		\item Sono stati fattorizzati moduli da 120 bit utilizzando l'algoritmo \emph{Pollard's rho} in poco meno di un'ora su un moderno calcolatore.
	\end{itemize}
\end{frame}

\begin{frame}{RSA: L'esponente pubblico $e$}
	\begin{itemize}
		\item L'esponente pubblico non dovrebbe essere troppo grande per velocizzare la cifratura.
		\item Con l'\emph{algoritmo delle quadrature successive}, l'operazione può essere svolta in tempo $O(\log_{2} e + hm(e))$, dove $hm(e)$ rappresenta il \emph{peso di Hamming}.
			\begin{itemize}
				\item Il peso di Hamming rappresenta il numero di simboli diversi dal simbolo $0$ dell'alfabeto utilizzato.
			\end{itemize}
		\item L'esponente pubblico, dato che non contiene alcuna informazione, viene generalmente riutilizzato per molteplici operazioni.
	\end{itemize}
\end{frame}

\begin{frame}{RSA: Valori più utilizzati di $e$ con i rispettivi \emph{pesi di Hamming}}
	\begin{table}[]
	\centering
	%\caption{Valori più usati di $e$ con i rispettivi \emph{Pesi di Hamming}~\cite{lenstra:publickeys}.}
\begin{tabular}{  M{1cm} 	M{0.5cm}  S[table-format=2.2, table-space-text-post = \si{\meter}] | M{1.3cm} M{0.5cm}  S[table-format=2.2, table-space-text-post = \si{\meter}]  | M{1cm} M{0.5cm} S[table-format=2.2, table-space-text-post = \si{\meter}] } 
	%\begin{tabular}{  cccccccccc } 
	\multicolumn{3}{c|}{\emph{X.509}} & \multicolumn{3}{c|}{\emph{PGP}} & \multicolumn{3}{c}{\emph{Combinati}} \\ \hline
	\multicolumn{1}{c}{$e$}           &\multicolumn{1}{c}{$hm(e)$} 	&\multicolumn{1}{c |}{$\%$}           & \multicolumn{1}{c}{$e$}   &\multicolumn{1}{c}{$hm(e)$}           & \multicolumn{1}{c |}{$\%$}      & \multicolumn{1}{c}{$e$}     &\multicolumn{1}{c}{$hm(e)$}       & \multicolumn{1}{c}{$\%$}             \\ \hline
	65537       & 2		& 98.4921     & 65537          	& 2		& 48.8501 	& 65537        	& 2			& 95.4933        \\
	17          	& 2		& 0.7633       & 17             		& 2		& 39.5027 	& 17           	& 2			& 3.1035         \\
	3           	& 2		& 0.3772       & 41             		& 3		& 7.5727  		& 41           	& 3			& 0.4574         \\
	35          	& 3		& 0.1410       & 19             		& 3		& 2.4774  		& 3            	& 2			& 0.3578         \\
	5           	& 2		& 0.1176       & 257            	& 2		& 0.3872  		& 19           	& 3			& 0.1506         \\
	7           	& 3		& 0.0631       & 23             		& 4		& 0.2212  		& 35           	& 3			& 0.1339         \\
	11          	& 3		& 0.0220       & 11             		& 3		& 0.1755  		& 5            	& 2			& 0.1111         \\
	47          	& 5		& 0.0101       & 3              		& 2		& 0.0565  		& 7            	& 3			& 0.0596         \\
	13          	& 3		& 0.0042       & 21             		& 3		& 0.0512  		& 11           	& 3			& 0.0313         \\
	65535       & 16		& 0.0011       & $2^{127} + 3$  	& 3		& 0.0248  		& 257          	& 2			& 0.0241         \\
	altri       	& -		& 0.0083       & altri          		& -		& 0.6807  		& altri        	& -			& 0.0774        
	\end{tabular}
	%\label{tabella:valorie}
	\end{table}

\end{frame}

\begin{frame}{RSA: Generazione errata della chiave}
	\begin{itemize}
		\item L'esponente $e$ deve essere scelto coprimo con $\phi(n)$.
		\item In una pre-release di Windows 10, non veniva effettuato il controllo che $gcd(e,\phi(n)) = 1$ nel momento in cui veniva scelto l'esponente pubblico.
		\item Il corretto funzionamento di RSA è compromesso.
	\end{itemize}
\end{frame}

\begin{frame}{RSA: la probabilità di scegliere l'esponente pubblico errato}
	\begin{itemize}
		\item Anche se non viene effettuato il controllo che $gcd(e,\phi(n)) = 1$, non è detto che l'uguaglianza non sia comunque verificata.
		\item 
	\end{itemize}
\end{frame}




















\end{document}
