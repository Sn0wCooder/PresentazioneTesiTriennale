\documentclass[11pt,svgnames,smaller,aspectratio=169]{beamer}

\usepackage[utf8]{inputenc}
\usepackage{amsmath}
\usepackage[italian]{babel}


\usetheme{UniPi}

% ------------------ CHANGE FROM HERE ------------------
\title{Crittografia a chiave pubblica: uno sguardo alle vulnerabilità \\ di RSA e Diffie-Hellman}
\author{Leonardo Alfreducci}
\institute[Università di Pisa]{Università di Pisa\\Dipartimento di Informatica}
\date{Pisa, 7 ottobre 2022}
% extra can be used for thesis supervisors and similar
\extra{Relatori\\Dott. Gaspare Ferraro\\Prof.ssa Anna Bernasconi}
% ------------------ END CHANGE ------------------

\usepackage{transparent}
\mode<presentation>

\begin{document}

\begin{frame} 
	\titlepage
\end{frame}
\logo{\transparent{0.2}\includegraphics[height=2cm]{images/cherubino}}



% ------------------ CHANGE FROM HERE ------------------
\begin{frame}{Table of Contents}
	\tableofcontents
\end{frame}

%%%%%%%%%%%%%%%%%%%%%%%%%%%%%%%%%%%%%%%%%%%%%%%%%%%%%%%%
\section{Introduzione}
\begin{frame}
	\sectionpage
	\centering
\end{frame}

\begin{frame}{Introduzione}
	\begin{itemize}
		\item Lo scambio di informazioni di ogni tipo avviene attraverso la rete: è dunque di fondamentale importanza proteggere le informazioni che vengono scambiate
		\item Si passeranno in rassegna i protocolli RSA e Diffie-Hellman (su campo primo e su curve ellittiche) 
	\end{itemize}
\end{frame}

\begin{frame}{Lorem ipsum}
	\begin{itemize}
		\item First item
			\begin{itemize}
				\item Sub item
			\end{itemize}
		\item Second item
		\item Third item
	\end{itemize}
\end{frame}

\begin{frame}{Lorem ipsum}
	\begin{enumerate}
		\item First item
		\item Second item
		\item Third item
	\end{enumerate}
	La complessità è $O(\sqrt(n))$
	
	\begin{equation}
	O(n)
	\end{equation}
\end{frame}

\begin{frame}[fragile]{Listings}
	\begin{lstlisting}
#include <stdio.h>

int main(void) {
	printf("Hello World\n");
	return 0;
}
	\end{lstlisting}
\end{frame}

\end{document}
